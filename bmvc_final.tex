\documentclass{bmvc2k}

%% Enter your paper number here for the review copy
% \bmvcreviewcopy{??}

\title{PythonRobotics: a Python code collection of robotics algorithms}

% Enter the paper's authors in order
% \addauthor{Name}{email/homepage}{INSTITUTION_CODE}
\addauthor{Atsushi Sakai}{https://atsushisakai.github.io/}{1}

% Enter the institutions
% \addinstitution{Name\\Address}
\addinstitution{
 University of California, Berkeley\\
 Berkeley, USA
}


\runninghead{arXiv}{Artificial Intelligence}

% Any macro definitions you would like to include
% These are not defined in the style file, because they don't begin
% with \bmva, so they might conflict with the user's own macros.
% The \bmvaOneDot macro adds a full stop unless there is one in the
% text already.
\def\eg{\emph{e.g}\bmvaOneDot}
\def\Eg{\emph{E.g}\bmvaOneDot}
\def\etal{\emph{et al}\bmvaOneDot}

%------------------------------------------------------------------------- 
% Document starts here
\begin{document}

\maketitle

\begin{abstract}
This paper describes an Open Source Software(OSS): PythonRobotics\cite{github}.This OSS is a Python code collection of robotics algorithms, especially focusing on autonomous navigation. It aims for beginners of robotics to understand basic ideas of each algorithm. The algorithms which is widely used in academia and industry and practical are selected. Each sample code only depends some standard modules on Python 3.x. In this paper, related works of this project, some key ideas about this OSS project, and brief structure of this repository are introduced. I also discuss future works of this project. 

\end{abstract}

%------------------------------------------------------------------------- 
\section{Introduction}


\section{Related works}

\section{Philosophy}
In this section, the philosophy of this project is described.
This project based on three philosophies.

The first one is that the codes have to be easy to read for understanding each algorithm's basic idea.
This project aims for beginners of robotics to understand basic ideas of each algorithm. 
Therefore, the code have to be easy to read and understand the algorithm.
Programming language, Python\cite{python} is adopted in this project because it has good code readability and it allows us to focus on algorithm itself.
Python has great libraries for matrix operation, mathematical and scientific operation, and visualization.
These libraries also allows us to focus on algorithm itself.

The second one is the algorithms which is widely used in academia and industry and practical are selected.
For example, Kalman filters and particle filter for localization, grid mapping for mapping, dynamic programming based approaches and sampling based approaches for path planning, and optimal control based approach for path tracking.

The third philosophy is minimum dependency.
Each sample code only depends some modules on Python3.

\begin{itemize}
 \item numpy\cite{numpy} for matrix operation
 \item scipy\cite{scipy} for mathematics, science, and engineering computing
 \item matplotlib\cite{matplotlib} for visualization
 \item pandas\cite{pandas} for data analysis
 \item cvxpy 0.4.x\cite{cvxpy} for convex optimization
\end{itemize}


\section{Repository structure}

\subsection{Localization}

\subsection{Mapping}

\subsection{SLAM}

\subsection{Path planning}

\subsection{Path tracking}


%------------------------------------------------------------------------- 
\section{Conclusion and future work}

In this paper, I introduced an OSS which is a Python code collection of robotics algorithms, especially for autonomous navigation. Related works of this project, some key ideas about this OSS project, and brief structure of this repository were described. 

The future works of this project is as followed: 

\begin{itemize}
 \item Technical and mathematical documentation with Jupyter notebook\cite{JupyterNotebook}.  
 \item Simple image processing samples for autonomous navigation only using OpenCV\cite{opencv}.
 \item Simple multi-robots simulations.
\end{itemize}

If readers were interested in these future projects, contributions are welcome.



%------------------------------------------------------------------------
\section{Acknowledgments}

I appreciate all contributors: Daniel Ingram\cite{auther1}, Joe Dinius\cite{auther2}, Karan Chala\cite{auther3}, Antonin RAFFIN\cite{auther4}, and Alexis Paques\cite{auther6}. This is my GitHub account\cite{auther5}

\bibliography{egbib}
\end{document}
