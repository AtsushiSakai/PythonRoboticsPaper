\documentclass{bmvc2k}
\usepackage{url}
\usepackage{hyperref}
\usepackage{soul}
\hypersetup{
    colorlinks=true,
    linkcolor=blue,
    filecolor=magenta,      
    urlcolor=cyan,
    pdftitle={PythonRobotics: a Python code collection of robotics algorithms},
    bookmarks=true,
    pdfpagemode=FullScreen,
}

%% Enter your paper number here for the review copy
% \bmvcreviewcopy{??}

\title{PythonRobotics: a Python code collection of robotics algorithms}

% Enter the paper's authors in order
% \addauthor{Name}{email/homepage}{INSTITUTION_CODE}
\addauthor{Atsushi Sakai}{https://atsushisakai.github.io/}{1}

\addauthor{Daniel Ingram}{ingramds@appstate.edu}{2}

\addauthor{Joseph Dinius}{jdinius@inviarobotics.com}{3}

\addauthor{Karan Chawla}{karan@skyryse.com}{4}

\addauthor{Antonin Raffin}{antonin.raffin@ensta-paristech.fr}{5}

\addauthor{Alexis Paques}{Alexis@unmanned.life}{6}


% Enter the institutions
% \addinstitution{Name\\Address}
\addinstitution{
 University of California, Berkeley
}

\addinstitution{
 Appalachian State University
}

\addinstitution{
 inVia Robotics, Inc.
}

\addinstitution{
 Skyryse Inc.
}

\addinstitution{
 ENSTA ParisTech
}

\addinstitution{
 Unmanned Life
}


\runninghead{arXiv}{Robotics, Artificial Intelligence}

% Any macro definitions you would like to include
% These are not defined in the style file, because they don't begin
% with \bmva, so they might conflict with the user's own macros.
% The \bmvaOneDot macro adds a full stop unless there is one in the
% text already.
\def\eg{\emph{e.g}\bmvaOneDot}
\def\Eg{\emph{E.g}\bmvaOneDot}
\def\etal{\emph{et al}\bmvaOneDot}
\def\jwd{\textcolor{red}}
%------------------------------------------------------------------------- 
% Document starts here
\begin{document}

\maketitle

\begin{abstract}
This paper describes an Open Source Software (OSS) project: PythonRobotics\cite{github}\st{.}\jwd{:}
\st{This is} a Python code collection of robotics algorithms\st{,}\jwd{.}  \jwd{The focus of the project} 
\st{especially focusing on} \jwd{is on} autonomous navigation\st{.}\jwd{,}
\st{It aims beginners of robotics to understand basic ideas of each algorithm.} \jwd{with the principle goal being to provide beginners with the tools necessary to understand it.}
In this project, the algorithms which are practical and widely used in both academia and industry are selected.
Each sample code is written in Python3\jwd{,} \st{and only depends on some standard modules for readability and ease of use} \jwd{has minimal dependencies,}  \st{It also provides} \jwd{and includes} intuitive animations \st{to understand the behavior of the simulation}.
%In this paper, related works of this project, some key ideas about this OSS project, and the brief structure of this repository are introduced.
%Future works of this project are also discussed. 

\end{abstract}

%------------------------------------------------------------------------- 
\section{Introduction}

In recent years, autonomous navigation technologies have received \st{a} huge attention in many fields. 
\st{For example,} \jwd{Such fields include:} autonomous driving\cite{pathplanning}, drone flight navigation, and other transportation systems.\jwd{INCLUDE ADDITIONAL REFERENCES IF YOU HAVE THEM}

An autonomous navigation system is \st{a system} \jwd{one} that can move \jwd{an object} to a goal\jwd{, potentially} \st{for} \jwd{over} long periods of time\jwd{,} without any external control by \jwd{a} human \jwd{operator}.
\st{The} \jwd{Such a} system requires a wide range of technologies\st{.} \jwd{:}
It needs to know where it is (localization), \st{where it is safe}\jwd{have knowledge of its environment} (mapping), \jwd{know} where and how to \st{go}\jwd{move} (path planning), and how to control \jwd{its motion with its} actuators (path following). 
The autonomous system would not work correctly if any of these technologies is missing.

Educational materials are becoming more and more important for future developers to learn basic autonomous navigation technologies.
\jwd{This is } \st{B}\jwd{b}ecause these \st{autonomous} technologies need different \st{technological}\jwd{technical skillsets} \st{backgrounds}\jwd{.} \st{such as} \jwd{Such skillsets include: }linear algebra, statistics, probability theory, optimization theory, control theory, and sensing physics\jwd{,} \st{etc.}\jwd{to name but a few.} 
\st{It needs a lot of time to be familiar with these technological areas.}\jwd{These skillsets can take a long time to learn, let alone become comfortable with.}
Therefore, good educational \st{materials}\jwd{resources} for learning basic autonomous navigation technologies are needed.  \jwd{PythonRobotics aims to be one such resource.}


\jwd{YOU PROBABLY DON'T WANT TO JUST REPEAT THE ABSTRACT.  THE FOLLOWING PARAGRAPH SHOULD BE REWRITTEN OR REMOVED.}
In this paper, an Open Source Software(OSS) project: PythonRobotics\cite{github} is described.
This project provides a code collection of robotics algorithms, especially focusing on autonomous navigation. 
It aims beginners of robotics to understand basic ideas of each algorithm.
It is written in Python\cite{python} under MIT license\cite{mit}.
It has a lot of simulation animations that shows behaviors of each algorithm.
It helps for learners to understand its fundamental ideas.
The readers can \st{check the}\jwd{review} animations in the project page \url{https://atsushisakai.github.io/PythonRobotics/}.
The left figure in Fig.\ref{fig:github} shows the front image of the page.
All source codes of this project are provided at \url{https://github.com/AtsushiSakai/PythonRobotics}.

This project was started from Mar. 21 2016 as a self-learning project.
It already has over 3000 stars on GitHub, and the right figure in Fig.\ref{fig:github} shows the history graph of the star counts using \cite{starhistory}.

\begin{figure}
\begin{tabular}{cc}
\bmvaHangBox{\fbox{\includegraphics[width=5.3cm]{images/projectpage.png}}}&
\bmvaHangBox{\fbox{\includegraphics[width=6.6cm]{images/starhistory.png}}}\\

\end{tabular}
\caption{Left: PythonRobotics Project page, Right: GitHub star history graph using \cite{starhistory}}
\label{fig:github}
\end{figure}

This paper is organized as follows: Section 2 reviews the related works. The philosophy of this project is presented in Section 3. \st{its}\jwd{The} repository structure and some \st{technological}\jwd{technical} backgrounds and simulation results are provided in Section 4. Conclusions \jwd{are drawn} and \st{some future works are drawn}\jwd{thoughts for future work are presented} in Section 5. 
Section 6 \st{shows acknowledgements for all}\jwd{acknowledges} contributors and \st{all} supporters. 


\section{Related works}

There are \jwd{already some} great educational materials for learning autonomous navigation technologies.

S. Thrun et al. wrote a great textbook\cite{PR} \st{"Probabilistic robotics"} which is a bible of localization, mapping, \jwd{and} Simultaneous Localization And Mapping (SLAM) for mobile robotics.
E. Frazzoli et al. wrote a great survey paper\cite{pathtracking} about path planning and control techniques for autonomous driving.
G. Bautista et al. wrote a survey paper\cite{pathplanning} focusing on path planning for automated vehicles.
J. Levinson wrote an overview paper\cite{Levinson2011} about systems and algorithms towards fully autonomous driving.

These papers help\st{s} readers to learn state\jwd{-}of\jwd{-}the\jwd{-}art autonomous navigation technologies.
However, it might be difficult for beginners to understand the basic ideas of the technologies and algorithms\st{,} because the\st{se} papers don't include implementation examples.

Many universities provide great classes to learn robotics and share \st{its}\jwd{their} lecture notes.
For example, University of Freiburg provides an introduction class \st{of}\jwd{to} mobile robotics\cite{course1}.
Swiss Federal Institute of Technology in Zurich (ETH in Zurich) also provides a class \st{of}\jwd{on} autonomous mobile robot\jwd{s}\cite{course2} and robotics programming with Robot Operating System(ROS)\cite{course3}.

\jwd{Like the academic literature referenced, } \st{T}\jwd{t}hese lecture notes also help\st{s} readers to learn \st{the} autonomous navigation technologies.
However, \st{it might be also difficult for beginners of robotics to understand the basic ideas of the technologies and algorithms if the reader is not a student of the class, because these lecture notes doesn't include implementation examples of robotics algorithms.}\jwd{, limited available examples of actual implementation of algorithms discussed might impede a beginner's ability to learn and use the material.}

Robot Operating System (ROS) is a middle-ware for robotics software development\cite{ros}\cite{rospaper}.
ROS initially released in 2007, \jwd{and} it \st{became a}\jwd{has become the} defacto standard platform in robotics software development.
ROS includes basic navigation software such as \jwd{the} Adaptive Monte Carlo Localization (AMCL) package and \jwd{the} local path planner based on \jwd{the Dynamic Window Approach}, etc\cite{rosnavigation}.
Many autonomous navigation packages using ROS are also \st{opened as an OSS}\jwd{open-sourced}.

\st{The }ROS packages \st{are}\jwd{can} also \jwd{be} useful to learn autonomous navigation algorithms\st{.}\jwd{,}
\st{H}\jwd{h}owever, \st{it might has an issue, because of lacking of documentation about its basic algorithm}\jwd{the documentation is inconsistent and, at times, it can be difficult to find implementation details about the algorithms used}.
\st{Those}\jwd{ROS} packages are usually written in C++ because the focus is computational efficiency and not readability, which makes learning from the code harder.

Udacity Inc, which is an online learning company founded by S. Thrun, et al, is providing great online courses \st{of}\jwd{for} autonomous navigation and autonomous driving \cite{udacity}.
\st{This}\jwd{These} course\jwd{s} provide\st{s} not only technical lectures, \jwd{but also} programming homework using Python.
\st{its}\jwd{This leads to a great} learning strategy \st{is great} for beginners to understand \st{its}\jwd{the} technical background \jwd{required for autonomous systems}.

\jwd{The} PythonRobotics project \jwd{seeks to extend Udacity's approach by giving beginners the necessary tools to understand and make further study into the methods of autonomous navigation.}\st{has a similar concept of the Udacity's course.
It aims beginners of robotics to understand basic technical ideas of each autonomous navigation algorithm using Python.}
The details of concepts of this OSS project will be described in the next section. 

\begin{figure}
\begin{tabular}{cc}
\bmvaHangBox{\fbox{\includegraphics[width=5.9cm]{images/localization1.png}}}&
\bmvaHangBox{\fbox{\includegraphics[width=6.2cm]{images/localization2.png}}}\\
Histogram filter localization&Particle filter localization
\end{tabular}
\caption{Localization simulation results}
\label{fig:localization}
\end{figure}

\section{Philosophy}
In this section, the philosophy of this project is described.
It based on three main philosophies.

\subsection{Readability}
The first one is that the code have to be easy to read for understanding each algorithm.
This project aims beginners of robotics to understand basic ideas of each algorithm. 
Python\cite{python} programming language is adopted in this project because of its simplicity and it allows us to focus on algorithm itself.
Python has great libraries for matrix operation, mathematical and scientific operation, and visualization.
These libraries also allows us to focus on algorithm itself.

\subsection{Practicality}
The second one is that implemented algorithms have to be practical and widely used in both academia and industry.
For example, Kalman filters and particle filter for localization, grid mapping for mapping, dynamic programming based approaches and sampling based approaches for path planning, and optimal control based approach for path tracking.

\subsection{Minimum dependency}
The last philosophy is minimum dependency.
It allows us to run code samples easily and to convert the Python codes to other programming languages such as C++, Java for practical usage.
Each sample code only depends some modules on Python3 as bellow.

\begin{itemize}
 \item numpy\cite{numpy} for matrix operation
 \item scipy\cite{scipy} for mathematics, science, and engineering computing
 \item matplotlib\cite{matplotlib} for visualization
 \item pandas\cite{pandas} for data import
 \item cvxpy\cite{cvxpy} for convex optimization
\end{itemize}

These modules are OSS and could be used for free.
This repository software doesn't depend on any commercial software.


\section{Repository structure}

In this section, we briefly described the repository structure.

This repository has five directories which corresponding to five technical categories in autonomous navigation: localization, mapping, SLAM, path planning, and path tracking.
There is one sub-directory per algorithm, which has a sample code using different algorithms.

In the following subsections, some algorithms and some simulation examples in each category are described.

\subsection{Localization}

Localization is the ability of a robot to know its position and orientation with sensors such as GNSS and IMU.
In localization, Bayesian filters such as Kalman filters, histogram filter, and particle filter are widely used\cite{PR}.
Fig.\ref{fig:localization} shows localization simulations using histogram filter and particle filter.



\subsection{Mapping}
Mapping is the ability of a robot to understand its surroundings with external sensors such as LIDAR and camera.
Robots have to recognize the position and shape of obstacles to avoid them.
In mapping, grid mapping and machine learning algorithms are widely used\cite{PR}\cite{PRML}.
Fig.\ref{fig:mapping} shows mapping simulation results using Grid mapping with 2D ray casting and 2D object clustering with k-means algorithm.

\begin{figure}
\begin{tabular}{cc}
\bmvaHangBox{\fbox{\includegraphics[width=6.0cm]{images/mapping1.png}}}&
\bmvaHangBox{\fbox{\includegraphics[width=5.7cm]{images/mapping2.png}}}\\
Grid mapping with 2D ray casting&2D object clustering with k-means algorithm
\end{tabular}
\caption{Mapping simulation results}
\label{fig:mapping}
\end{figure}

\subsection{SLAM}
Simultaneous Localization and Mapping (SLAM) is an ability to estimate the pose of a robot and the map of the environment at the same time.
SLAM problem is hard to solve, because a map is needed for localization and a good localization is needed for mapping, which is a kind of chicken and egg problem.
Popular SLAM solution methods include the extended Kalman filter, particle filter, and Fast SLAM algorithm\cite{PR}.
Fig.\ref{fig:slam} shows SLAM simulation results using Extended Kalman Filter and results using FastSLAM2.0\cite{PR}.

\begin{figure}
\begin{tabular}{cc}
\bmvaHangBox{\fbox{\includegraphics[width=6.0cm]{images/slam1.png}}}&
\bmvaHangBox{\fbox{\includegraphics[width=6.0cm]{images/slam2.png}}}\\
Extended Kalman Filter based SLAM & FastSLAM 2.0 based SLAM
\end{tabular}
\caption{SLAM simulation results}
\label{fig:slam}
\end{figure}

\subsection{Path planning}

Path planning is the ability of a robot to search feasible and efficient path to the goal.
The path have to satisfy some constraints based on movement model and obstacle positions, and optimize some objective functions such as time to goal and distance to obstacle etc.
In path planning, dynamic programming based approaches and sampling based approaches are widely used\cite{pathplanning}.
Fig.\ref{fig:pathplan} shows simulation results of potential field path planning and LQR-RRT* path planning\cite{lqrrrt}.

\begin{figure}
\begin{tabular}{cc}
\bmvaHangBox{\fbox{\includegraphics[width=6.0cm]{images/pathplan1.png}}}&
\bmvaHangBox{\fbox{\includegraphics[width=6.0cm]{images/pathplan2.png}}}\\
Potential Field path planning & LQR-RRT* path planning\cite{lqrrrt}
\end{tabular}
\caption{Path planning simulation results}
\label{fig:pathplan}
\end{figure}

\subsection{Path tracking}
Path tracking is an ability of a robot to follow the reference path generated by path planning algorithms.
The role of the path tracking controller is to stabilize to the reference path or trajectory which has modeling error and other forms of uncertainty. 
In path tracking, feedback control techniques and optimization based control techniques are widely used\cite{pathplanning}.
Fig.\ref{fig:pathtracking} shows simulations using rear wheel feedback steering control and PID speed control, and iterative linear model predictive path tracking control\cite{lqrrrt}.

\begin{figure}
\begin{tabular}{cc}
\bmvaHangBox{\fbox{\includegraphics[width=6.0cm]{images/path_tracking1.png}}}&
\bmvaHangBox{\fbox{\includegraphics[width=6.2cm]{images/path_tracking2.png}}}\\
Rear wheel feedback steering control \\ and PID speed control \cite{pathtracking}& Iterative linear model predictive control
\end{tabular}
\caption{Path tracking simulation results}
\label{fig:pathtracking}
\end{figure}

%------------------------------------------------------------------------- 
\section{Conclusion and future work}

In this paper, we introduced PythonRobotics, a code collection of robotics algorithms, with a focus on autonomous navigation. especially for autonomous navigation.
Related works of this project, some key ideas about this OSS project, and brief structure of this repository and some simulation results were described. 

The future works of this project are as followed: 

\begin{itemize}
 \item Adding technical and mathematical documentation using Jupyter notebook\cite{JupyterNotebook}.  
 \item Adding simple image processing samples for autonomous navigation only using OpenCV\cite{opencv}.
 \item Adding simple multi-robots simulations for autonomous navigation.
\end{itemize}

If readers were interested in these future projects, contributions are welcome.
Readers can also support this project financially via Patreon\cite{patreon}.

%------------------------------------------------------------------------
\section{Acknowledgments}

We appreciate all contributors: Atsushi Sakai\cite{auther5}, Daniel Ingram\cite{auther1}, Joe Dinius\cite{auther2}, Karan Chala\cite{auther3}, Antonin RAFFIN\cite{auther4}, and Alexis Paques\cite{auther6}.

We also appreciate all robotics lovers who give a star to this repository in GitHub.


\bibliography{egbib}
\end{document}
